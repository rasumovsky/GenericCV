%%%%%%%%%%%%%%%%%%%%%%%%%%%%%%%%%%%%%%%%%%%%%%%%%%%%%%%%%%%%%%%%%%%%%%%%%%%%%%%%
%
% Resume
% Author: Andrew Hard
% Date: August 19, 2016
%
%%%%%%%%%%%%%%%%%%%%%%%%%%%%%%%%%%%%%%%%%%%%%%%%%%%%%%%%%%%%%%%%%%%%%%%%%%%%%%%%

\documentclass{letter}

\usepackage{amssymb} % for black square
\usepackage[usenames, dvipsnames]{color}
%\definecolor{Maroon}{cmyk}{0, 0.87, 0.68, 0.32}
\definecolor{Maroon}{cmyk}{0, 0.9, 0.8, 0.55}
\usepackage{geometry} % for paper geometry
\usepackage{graphicx}
\usepackage{courier}
\usepackage{hyperref} % for hyperlinks
\hypersetup{hidelinks}
%\hypersetup{colorlinks=true,urlcolor=blue}
%\usepackage{helvetica}
\usepackage{newcent}
% Set the paper size and margin size (use a4paper, not legalpaper):
\geometry{a4paper, portrait, margin=0.5in}
\begin{document} 

\pagenumbering{gobble}

%%%%%%%%%%%%%%%%%%%%%%%%%%%%%%%%%%%%%%%%%%%%%%%%%%%%%%%%%%%%%%%%%%%%%%%%%%%%%%%%
\begin{tabular}{p{0.45\textwidth}p{0.55\textwidth}}
	\hfill \newline \href{https://ch.linkedin.com/in/andrew-hard-25b690a5}{\Huge{\color{Maroon}{Andrew Hard}}} \newline \LARGE{\textit{R\'{e}sum\'{e}}} \newline
	&
	\hfill \newline CERN PH Division, B32 RA-14, 1211 Gen\`{e}ve, Switzerland \newline
	(+41) 76 30 88 007, (+1) 423 227 4106 \newline
	\href{mailto:ahard@cern.ch}{andrew.straiton.hard@cern.ch} \newline
	\href{https://github.com/rasumovsky}{github.com/rasumovsky}\\
\end{tabular}

%%%%%%%%%%%%%%%%%%%%%%%%%%%%%%%%%%%%%%%%%%%%%%%%%%%%%%%%%%%%%%%%%%%%%%%%%%%%%%%%
\begin{flushleft}
\Large{\textsc{\textbf{\color{Maroon}{Education}}}}
\vspace{1pt} % space between text and line
\hrule
\end{flushleft}

\begin{tabular}{p{0.15\textwidth}p{0.8\textwidth}}
	2016 (Expected)
	&
	\textbf{Doctor of Philosophy} in Physics \newline 
	\textit{University of Wisconsin}, Madison WI, USA \newline
	Thesis: \textit{Search and discovery with the resonant $\gamma\gamma$ final state at ATLAS} \newline
	Advised by Prof. Sau Lan Wu \\
\\
	2010 
	& 
	\textbf{Bachelor of Arts} in Physics, Honors \newline 
	\textit{University of Chicago}, Chicago IL, USA \newline
	Advised by Prof. Edward Blucher 
\\
\end{tabular}

%%%%%%%%%%%%%%%%%%%%%%%%%%%%%%%%%%%%%%%%%%%%%%%%%%%%%%%%%%%%%%%%%%%%%%%%%%%%%%%%
\begin{flushleft}
\Large{\textsc{\textbf{\color{Maroon}{Experience}}}}
\vspace{1pt} % space between text and line
\hrule
\end{flushleft}

%\renewcommand{\arraystretch}{0.5}% Wider
\begin{tabular}{p{0.15\textwidth}p{0.8\textwidth}}
	2011 - 2016
	&
	\textbf{Graduate Research Assistant}, \textit{Department of Physics, University of Wisconsin} \newline
	\vspace{-15pt}      % suppress blank line after tabbing
	\begin{itemize}
		\itemsep0em
%		\renewcommand{\labelitemi}{\scriptsize$\blacksquare$} 
		\renewcommand{\labelitemi}{\tiny$\blacksquare$}
    		\item Discovered Higgs boson, performed first measurements of mass, couplings, and spin 
		\item Contributed significantly to 20 papers \& notes, author on 250+ ATLAS publications
    		\item Statistical expert, created new Monte Carlo method to reduce CPU usage by $1000\times$
		\item Invented algorithm to spatially and temporally match CMOS chip hits at LBNL 
    		\item Developed analysis software with C++, ROOT, \& shell scripts for ATLAS collaboration
		\item Optimized physics searches with massive datasets using machine learning techniques
		\item Wrote and coordinated successful DoE funding reports for Wisconsin ATLAS Group
	\end{itemize}
\\
	2014 
	& 
	\textbf{Graduate Teaching Assistant}, \textit{Department of Physics, University of Wisconsin} \newline
	\vspace{-15pt}      % suppress blank line after tabbing
	\begin{itemize}
		\itemsep0em
		\renewcommand{\labelitemi}{\tiny$\blacksquare$} 
		\item Led discussions and labs on classical mechanics, electrodynamics, thermodynamics
    		\item Designed supplemental exercises and summary notes that boosted exam performances
	\end{itemize}
\\
	2010 - 2011
	&
	\textbf{CERN Technologist}, \textit{Enrico Fermi Institute, University of Chicago} \newline
	\vspace{-15pt}      % suppress blank line after tabbing
	\begin{itemize}
		\itemsep0em
		\renewcommand{\labelitemi}{\tiny$\blacksquare$} 
   		\item Electronic calibration expert for the ATLAS Experiment hadronic calorimeter
   		\item Developed \& maintained calibration software package using Python and MySQL
    		\item Documented, monitored and reported on detector status to collaboration
	\end{itemize}
\\
	2009 - 2010
	&
	\textbf{Undergraduate Research Assistant}, \textit{Enrico Fermi Institute, University of Chicago} \newline
	\vspace{-15pt}      % suppress blank line after tabbing
	\begin{itemize}
		\itemsep0em
		\renewcommand{\labelitemi}{\tiny$\blacksquare$} 
   		\item Developed particle detector simulation in C++ with ROOT and Geant4 libraries
   		\item Constructed $\mu$ particle modules, worked in machine shop, tested electronics 
	\end{itemize}
\end{tabular}

\vspace{-10pt}

%%%%%%%%%%%%%%%%%%%%%%%%%%%%%%%%%%%%%%%%%%%%%%%%%%%%%%%%%%%%%%%%%%%%%%%%%%%%%%%%
\begin{flushleft}
\Large{\textsc{\textbf{\color{Maroon}{Skills}}}}
\vspace{1pt} % space between text and line
\hrule
\end{flushleft}

\begin{tabular}{p{0.15\textwidth}p{0.8\textwidth}}
	{\bf Scientific} 
	&
	Physics, Statistics, Simulation, Numerical Methods, Data Structures, High Throughput Computing, Databases, Machine Learning, Public Presentation \newline
\\
	{\bf Programming} 
	&
	C++, Python, Java, \LaTeX{}, Unix/Linux shell scripting, ROOT, Matlab, SQL, TensorFlow \newline
\\ 
	{\bf Languages}
	&
	English (native), French (basic oral and written communication)
\end{tabular}

%%%%%%%%%%%%%%%%%%%%%%%%%%%%%%%%%%%%%%%%%%%%%%%%%%%%%%%%%%%%%%%%%%%%%%%%%%%%%%%%
\begin{flushleft}
\Large{\textsc{\textbf{\color{Maroon}{Volunteering \& Outreach}}}}
\vspace{1pt} % space between text and line
\hrule
\end{flushleft}
	\vspace{-10pt}      % suppress blank line after tabbing
\begin{tabular}{p{0.15\textwidth}p{0.8\textwidth}}
	\href{https://twiki.cern.ch/twiki/pub/AtlasPublic/HiggsPublicResults//Hgg-FloatingScale-Short2.gif}{\raisebox{-0.95\height}{\includegraphics[scale=0.16]{AnimationQR.jpg}}}
	&
	\begin{itemize}
		\itemsep0em
		\renewcommand{\labelitemi}{\tiny$\blacksquare$} 
		\item Newtonian physics demonstration for Chicago Public Library \hfill 2016
		\item US voter outreach \& registration at CERN \hfill 2016
		\item Discussed research \& funding with U.S. lawmakers in Washington D.C. \hfill 2014, 2015
		\item Created GIF visualizations of Higgs boson discovery data \hfill 2013
		\item Visited classrooms at the Chattanooga School for the Arts \& Sciences \hfill 2012
	\end{itemize}	
\end{tabular}

\vspace{-10pt}

%%%%%%%%%%%%%%%%%%%%%%%%%%%%%%%%%%%%%%%%%%%%%%%%%%%%%%%%%%%%%%%%%%%%%%%%%%%%%%%%
\begin{flushleft}
\Large{\textsc{\textbf{\color{Maroon}{Awards}}}}
\vspace{1pt} % space between text and line
\hrule
\end{flushleft}

\begin{tabular}{p{0.15\textwidth}p{0.8\textwidth}}
	2015
	&
	\textbf{Teaching Assistant Rookie of the Year}, \textit{Department of Physics, University of Wisconsin} 
\\
	2013, 2014
	& 
	\textbf{Lightning Round Winner}, \textit{US LHC User's Association Annual Meeting}
\\
\end{tabular}

\end{document}